\begin{center}
	\begin{tabular}{C{0.35cm}C{1.3cm}C{1cm}C{1cm}C{1cm}C{1cm}C{1cm}}
		$N$ & $m$      & $t_1$   & $t_2$   & $t_\Delta $ & $T$    & $g$    \\
		\bottomrule
		1   & $m_i$    & $17.45$ & $17.40$ & $17.43$     & $1.74$ & $9.77$ \\
		2   & $m_1+2$  & $17.41$ & $17.38$ & $17.4$      & $1.74$ & $9.77$ \\
		3   & $m_1+5$  & $17.45$ & $17.55$ & $17.5$      & $1.75$ & $9.66$ \\
		4   & $m_1+10$ & $17.42$ & $17.53$ & $17.48$     & $1.75$ & $9.66$ \\
		5   & $m_1+20$ & $17.42$ & $17.50$ & $17.46$     & $1.75$ & $9.66$
	\end{tabular}
\end{center}

\begin{center}
	\begin{tabular}{C{0.5cm}C{1cm}C{1cm}C{1cm}C{1cm}C{1cm}C{1cm}}
		$N$ & $L$  & $t_1$   & $t_2$   & $t_\Delta$ & $T$    & $g$   \\
		\bottomrule
		1   & $75$ & $17.43$ & $17.58$ & $17.51$    & $1.75$ & $9.7$ \\
		2   & $60$ & $15.51$ & $15.60$ & $15.56$    & $1.56$ & $9.7$ \\
		3   & $45$ & $13.55$ & $13.47$ & $13.51$    & $1.35$ & $9.7$ \\
		4   & $30$ & $11.15$ & $11.06$ & $11.11$    & $1.11$ & $9.6$ \\
		5   & $15$ & $ 7.95$ & $ 7.64$ & $ 7.8$     & $0.78$ & $9.7$
	\end{tabular}
\end{center}
$====================================================$
\begin{center}
	\begin{tabular}{C{0.5cm}C{1cm}C{1cm}C{1cm}C{1cm}C{1cm}C{1cm}}
		$N$ & $m$   & $t_1$  & $t_2$  & $t_\Delta$ & $T$    & $K$    \\
		\bottomrule
		1   & $0.4$ & $4.58$ & $4.53$ & $4.56$     & $0.46$ & $7.61$ \\
		2   & $0.7$ & $6.02$ & $6.03$ & $6.03$     & $0.60$ & $7.61$ \\
		3   & $0.1$ & $7.17$ & $7.23$ & $7.20$     & $0.72$ & $7.62$ \\
		4   & $0.13$ & $8.24$ & $8.18$ & $8.21$     & $0.82$ & $7.61$ \\
		5   & $0.16$ & $9.09$ & $9.13$ & $9.11$     & $0.91$ & $7.61$
	\end{tabular}
\end{center}

\begin{center}
	\begin{tabular}{C{0.5cm}C{1cm}C{1cm}C{1cm}C{1cm}}
		$N$ & $F$    & $x_i$  & $x_f$   & $K$    \\
		\bottomrule
		1   & $0.04$ & $10.6$ & $15.75$ & $7.61$ \\
		2   & $0.07$ & $10.6$ & $19.6$  & $7.62$ \\
		3   & $0.1$  & $10.6$ & $23.46$ & $7.62$ \\
		4   & $0.13$ & $10.6$ & $27.33$ & $7.62$ \\
		5   & $0.16$ & $10.6$ & $31.2$  & $7.61$
	\end{tabular}
\end{center}

\section{Procedimiento Experimental}
%\begin{dyNoteImportant}[morado01!20]{azulfor!10}{black!80}{Procedimientos}
%	\begin{itemize}[label=\textbf{$\bullet$},itemsep=2pt,partopsep=6pt]
%		\item Arme el equipo como se muestra en la figura.
%		\item Calienta la miuestra de masa $m$ y calor específico $c$ a $T_1$, sumergido en el agua en ebullición.
%		\item Sumerge la muestra en agua fría de masa $M$ que contiene un calorímetro de masa $M'$ y de calor específico $c'$.
%		\item Controle la temperatura $T_2$ del agua fría en el momento en que se va introducir la muestra.
%		\item Agite constantemente el sistema hasta observar la temperatura del ambiente y anote los datos en la tabla.
%		\item Haciendo un balance de calor ganado igual al calor perdido, calcule el calor específico de la muestra.
%		\item Repita los pasos anteriores para otras muestras.
%	\end{itemize}
%\end{dyNoteImportant}
\subsection{Para el Agua}
\begin{enumerate}[label=\bfseries\alph*.-,itemsep=2pt, partopsep=6pt]
	\item \textbf{Calculando el equivalente en agua del calorímetro}
	      \[k=\cfrac{m(T-T_e)}{T_e-T_0}-M\]
	      Reemplzando:
	      \[k=\cfrac{200(91-54)}{54-18.5}-120\]
	      Finalmente:
	      \[k=88.54\]
	\item \textbf{Calculando el calor específico del sólido}
	      \[c=\cfrac{(M+k)(T_e-T_0)}{m\cdot(T-T_e)}\]
	      Reemplzando:
	      \[c=\cfrac{(120+k)(54-18.5)}{200\times(91-54)}\]
	      Finalmente:
	      \[c=1.000\,\,4\]
\end{enumerate}
%-------------------------------------------------------------------------------------------
\subsection{Para el Aluminio}
\begin{enumerate}[label=\bfseries\alph*.-,itemsep=2pt, partopsep=6pt]
	\item \textbf{Calculando el equivalente en agua del calorímetro}
	      \[k=\cfrac{m(T-T_e)}{T_e-T_0}-M\]
	      Reemplzando:
	      \[k=\cfrac{11.75(93-21.5)}{21.5-19.5}-100\]
	      Finalmente:
	      \[k=0.320\]
	\item \textbf{Calculando el calor específico del sólido}
	      \[c=\cfrac{(M+k)(T_e-T_0)}{m\cdot(T-T_e)}\]
	      Reemplzando:
	      \[c=\cfrac{(100+0.32)(21.5-19.5)}{11.75\times(93-21.5)}\]
	      Finalmente:
	      \[c=0.23\]
\end{enumerate}
%-------------------------------------------------------------------------------------------
\subsection{Para el Hierro}
\begin{enumerate}[label=\bfseries\alph*.-,itemsep=2pt, partopsep=6pt]
	\item \textbf{Calculando el equivalente en agua del calorímetro}
	      \[k=\cfrac{m(T-T_e)}{T_e-T_0}-M\]
	      Reemplzando:
	      \[k=\cfrac{32.9(95-22.5)}{22.5-18.5}-100\]
	      Finalmente:
	      \[k=0.496\]
	\item \textbf{Calculando el calor específico del sólido}
	      \[c=\cfrac{(M+k)(T_e-T_0)}{m\cdot(T-T_e)}\]
	      Reemplzando:
	      \[c=\cfrac{(100+0.496)(22.5-18.5)}{32.9\times(95-18.5)}\]
	      Finalmente:
	      \[c=0.12\]
\end{enumerate}
%-------------------------------------------------------------------------------------------
\subsection{Para el Cobre}
\begin{enumerate}[label=\bfseries\alph*.-,itemsep=2pt, partopsep=6pt]
	\item \textbf{Calculando el equivalente en agua del calorímetro}
	      \[k=\cfrac{m(T-T_e)}{T_e-T_0}-M\]
	      Reemplzando:
	      \[k=\cfrac{37(94-21)}{21-19.5}-100\]
	      Finalmente:
	      \[k=0.018\]
	\item \textbf{Calculando el calor específico del sólido}
	      \[c=\cfrac{(M+k)(T_e-T_0)}{m\cdot(T-T_e)}\]
	      Reemplzando:
	      \[c=\cfrac{(100+0.018)(21-19.5)}{37\times(94-21)}\]
	      Finalmente:
	      \[c=0.090\]
\end{enumerate}
%==========================================================================================
\section{Tabla y Resultados}
\subsection{Para el Agua}
\begin{table}[H]%*******************AGUA****************
	\centering
	\begin{tabular}{L{2cm}C{1.5cm}C{2.2cm}C{2cm}C{1.3cm}C{2.5cm}}
		\rowcolor{gray!10} & Masa    & Temp. Inicial & Temp. Final & $\Delta T$ & Calor Específico \\
		Agua Fría          & $150$   & $18.5$        & $54$        & $35.5$     & $1$              \\
		Calorímetro        & $122.1$ & $18.5$        & $54$        & $35.5$     & $1$              \\
		Muestra            & $200$   & $91$          & $34$        & $57$       & $1.0004$         \\
	\end{tabular}
\end{table}
%-------------------------------------------------------------------------------------------
\subsection{Para el Aluminio}
\begin{table}[H]%*******************ALUMINIO****************
	\centering
	\begin{tabular}{L{2cm}C{1.5cm}C{2.2cm}C{2cm}C{1.3cm}C{2.5cm}}
		\rowcolor{gray!10} & Masa    & Temp. Inicial & Temp. Final & $\Delta T$ & Calor Específico \\
		Agua Fría          & $100$   & $19.5$        & $21.5$      & $2$        & $1$              \\
		Calorímetro        & $122.1$ & $19.5$        & $21.5$      & $2$        & $1$              \\
		Muestra            & $11.75$ & $93$          & $21.5$      & $71.5$     & $0.23$           \\
	\end{tabular}
\end{table}
%-------------------------------------------------------------------------------------------
\subsection{Para el Hierro}
\begin{table}[H]%*******************FIERRO****************
	\centering
	\begin{tabular}{L{2cm}C{1.5cm}C{2.2cm}C{2cm}C{1.3cm}C{2.5cm}}
		\rowcolor{gray!10} & Masa    & Temp. Inicial & Temp. Final & $\Delta T$ & Calor Específico \\
		Agua Fría          & $100$   & $18.5$        & $22.5$      & $4$        & $1$              \\
		Calorímetro        & $122.1$ & $18.5$        & $22.5$      & $4$        & $1$              \\
		Muestra            & $32.9$  & $95$          & $22.5$      & $72.5$     & $0.12$           \\
	\end{tabular}
\end{table}
%-------------------------------------------------------------------------------------------
\subsection{Para el Cobre}
\begin{table}[H]%*******************COBRE****************
	\centering
	\begin{tabular}{L{2cm}C{1.5cm}C{2.2cm}C{2cm}C{1.3cm}C{2.5cm}}
		\rowcolor{gray!10} & Masa    & Temp. Inicial & Temp. Final & $\Delta T$ & Calor Específico \\
		Agua Fría          & $100$   & $19.5$        & $21$        & $1.5$      & $1$              \\
		Calorímetro        & $122.1$ & $19.5$        & $21$        & $1.5$      & $1$              \\
		Muestra            & $37$    & $94$          & $21$        & $0$        & $0.09$           \\
	\end{tabular}
\end{table}
%==========================================================================================
\section{Cuestionario}
\begin{enumerate}[itemsep=2pt]
	\item Determine le calor específico y calcule el error de incertidumbre.\\
	      \textbf{Calculando las incertidumbres}
	      \[E_r \%= \cfrac{\left\lvert V_T-V_R\right\rvert }{V_T}\times100\]

	      %\sdconditions[12cm]{azzul}{Donde:}{%
		    %  \begin{tabular}{lcl}
			  %    $E_r$ & \@: & Error porcentual o incertidumbre. \\
			  %    $V_T$ & \@: & Valor Teórico o Bibliográfico.    \\
			  %    $V_R$ & \@: & Valor Experiemntal u obtenido.
		    %  \end{tabular}
	      %}
	      \begin{itemize}[label=\textbf{$\bullet$},itemsep=2pt,partopsep=6pt]
		      \item Para la densidad del agua.
		            \[E_r \%= \cfrac{\left\lvert 1-1.004\right\rvert }{1}\times100=0.4\]
		      \item Para la densidad del aluminio ($Al$).
		            \[E_r \%= \cfrac{\left\lvert 0.22-0.23\right\rvert }{0.22}\times100=4.54\]
		      \item Para la densidad del hierro ($Fe$).
		            \[E_r \%= \cfrac{\left\lvert 0.11-0.12\right\rvert }{0.11}\times100=9.09\]
		      \item Para la densidad del cobre ($Cu$).
		            \[E_r \%= \cfrac{\left\lvert 0.093-0.090\right\rvert }{0.093}\times100=3.23\]
	      \end{itemize}
	\item Determine el calor específico de cada una de las muestras y compare con el valor teórico.
	      \begin{table}[H]
		      \centering
		      \begin{tabular}{L{2cm}C{3cm}C{3cm}}
			      \rowcolor{gray!10}Muestra & Valor Teórico & Valor Esperimental \\
			      Agua                      & $1$           & $1.0004$           \\
			      Aluminio                  & $0.11$        & $0.12$             \\
			      Hierro                    & $0.22$        & $0.23$             \\
			      Cobre                     & $0.093$       & $0.09$             \\
		      \end{tabular}
	      \end{table}
	\item ¿Por qué una persona se pone chompa ``para protegerse del frío''? y ¿por qué no se pone la chompa para protegerse del calor? Explique qué debe ser correcto.

	      Una persona se pone la chompa porque es una aislante térmico que 	mantiene el calor en nuestro cuerpo con el medio ambiente ya que si no se 	utilizaría 	la chompa del frió habría una continua 	transmisión 	de calor de nuestro  hacia el medio circundante, por tanto el primero es 	lo correcto
\end{enumerate}
%========================================================================================== 
%\cleardoublepage%
%\section*{Bibliografía}
%$\left[1\right]$ Sears, F.W., Zemansky, M.W., Young, H.D., Freedman, R.A. (2013). Física Universitaria. Volumen I. Décimo tercera edición. México: Pearson Education.
%
%$\left[2\right]$ Resnick, R., Halliday, D., Krane, K. (2013). Física. Volumen I. Quinta edición. México: Grupo Editorial Patria.
%
%$\left[3\right]$ Serway, R.A. y Jewett, J.W. (2008). Física Para Ciencias e Ingeniería. Volumen I. Sétima edición. México: Cengage Learning Editores S.A. de C.V.
%
%$\left[4\right]$ Wilson, J.D., Buffa, A.J. y Lou, B. (2007). Física. Sexta edición. México:Pearson educación.