\section{Antecedentes}

En esta subsección, se presentan los antecedentes relevantes que contextualizan el tema de investigación. Se revisan estudios previos, investigaciones relacionadas y desarrollos relevantes en el área de interés. Los antecedentes ayudan a establecer la continuidad y evolución del conocimiento en el campo, proporcionando un punto de partida para el desarrollo del marco teórico y la investigación posterior.

Los antecedentes pueden incluir una revisión histórica del tema, identificación de problemas previamente abordados, descripción de avances tecnológicos o científicos relevantes, y un resumen de las investigaciones clave realizadas por otros investigadores en el área.

\section{Marco Teórico}

El marco teórico proporciona el contexto conceptual y la base teórica sobre la cual se fundamenta el trabajo realizado. En esta sección, se revisan y analizan las teorías, conceptos y modelos relevantes relacionados con el tema de investigación. El objetivo es establecer una comprensión sólida de los principios fundamentales que guían el estudio y proporcionar un fundamento sólido para la interpretación de los resultados.

En el contexto de este documento, se explorarán diversas teorías y enfoques relacionados con [insertar aquí el tema específico de tu investigación]. Se abordarán conceptos clave como [mencionar algunos conceptos relevantes] y se discutirá su importancia en el contexto del problema de investigación planteado.

Es importante destacar que el marco teórico no solo debe presentar la información de manera descriptiva, sino que también debe analizar críticamente las diferentes perspectivas y enfoques, identificando sus fortalezas, debilidades y posibles implicaciones para el estudio en cuestión.

Al final de esta sección, los lectores deben tener una comprensión clara del contexto teórico en el que se inscribe el trabajo y cómo se relaciona con el objetivo general de la investigación.

\newpage