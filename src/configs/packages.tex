\RequirePackage{rotating}
\RequirePackage{pdflscape}
\RequirePackage{bookmark}
\RequirePackage{tikz}
\usetikzlibrary{babel}
\RequirePackage[export]{adjustbox}
\usetikzlibrary{shadows}
\RequirePackage{booktabs}
\RequirePackage{longtable}
\RequirePackage{multirow}
\RequirePackage{multicol}
\setlength\columnseprule{0.5pt}
\RequirePackage{array}
\RequirePackage{tabularx}
\RequirePackage[rflt]{floatflt}
\RequirePackage{enumitem}
\setlist[itemize]{itemsep=-1.5mm,topsep=-5pt,partopsep=9pt}
\setlist[description]{itemsep=-1.5mm,topsep=-1pt,parsep=5pt,partopsep=9pt}
\setlist[enumerate]{itemsep=-1.5mm,topsep=-5pt,partopsep=9pt}
\setlist[itemize,1]{label=\leavevmode\hbox{} to 1.2ex{\hss\vrule height .9ex width .7ex depth -.2ex\hss}}

\RequirePackage{listings}
\RequirePackage{listingsutf8}
\newcommand{\thess}{%
\renewcommand{\contentsname}{Índice general}
\renewcommand{\partname}{Parte}
\renewcommand{\appendixname}{Anexo}
\renewcommand{\figurename}{Figura}
\renewcommand{\tablename}{Tabla}
\renewcommand{\lstlistingname}{Código}
\renewcommand{\chaptername}{Capítulo}
\renewcommand{\bibname}{Referencias bibliográficas}
\renewcommand\indexname{Índice alfabético}
\renewcommand\listfigurename{Índice de figuras}
\renewcommand\listtablename{Índice de tablas}
\renewcommand\lstlistlistingname{Índice de códigos}
\gdef\thelstlisting{\arabic{section}.\arabic{lstlisting}}
}

%----------------------------------------------------------------------------------------------------

\RequirePackage{blkarray}
\RequirePackage{makecell}
\RequirePackage{cellspace}
\RequirePackage{hhline}
\RequirePackage{color}
\RequirePackage{environ}
\RequirePackage{pdfpages}
\RequirePackage{float}
\RequirePackage{subfig}
\RequirePackage{graphicx}
\RequirePackage{wrapfig}
\usepackage{chngcntr}
\counterwithout{table}{section}
\counterwithout{figure}{section}
\numberwithin{figure}{section}

%----------------------------------------------------------------------------------------------------

\DeclareCaptionLabelSeparator{syfsepAPAt}{\quad}
\DeclareCaptionFont{cfblack}{\color{black}}
\DeclareCaptionFont{sybf}{\fontsize{10pt}{14pt}\selectfont\bfseries}
\DeclareCaptionFont{synfapat}{\fontsize{10pt}{14pt}\itshape\selectfont}%
\DeclareCaptionFormat{sybformat}%
{#1#2\par\vspace*{-6pt}\parbox[t]{\linewidth}{\itshape#3}}%
\usepackage{caption}
\captionsetup{format=hang,justification=justified}
\captionsetup[lstlisting]{labelfont={cfblack,bf},textfont={cfblack},skip=0pt,aboveskip=5pt,belowskip=0pt}
\captionsetup[listing]{labelfont={cfblack,bf},textfont={cfblack},skip=0pt,aboveskip=0pt,belowskip=0pt}
\captionsetup[figure]{format=sybformat,margin=0mm,skip=6pt,labelsep=syfsepAPAt,labelfont={cfblack,sybf},textfont={cfblack,synfapat},indention=0pt,belowskip=0pt}
\captionsetup[table]{format=sybformat,margin=0mm,skip=6pt,labelsep=syfsepAPAt,labelfont={cfblack,sybf},textfont={cfblack,synfapat},indention=0pt,belowskip=0pt}
%=================================
\DeclareCaptionFormat{syapafformat}{#1#2\,#3}
\DeclareCaptionLabelSeparator{syfsep}{:\space}
\captionsetup[figure]{format=syapafformat,margin=0mm,skip=6pt,labelsep=syfsep,labelfont={cfblack,sybf},textfont={cfblack},indention=0pt,belowskip=0pt,name={Fig.}}
%=================================
\DeclareCaptionFormat{syapafformatz}{#1#2\,\,#3}
\DeclareCaptionLabelSeparator{syfsepAPAf}{:}
\DeclareCaptionFont{synf}{\normalfont\selectfont}
\captionsetup[figure]{format=syapafformatz,margin=0mm,skip=6pt,labelsep=syfsepAPAf,labelfont={cfblack,sybf},textfont={cfblack,synf},indention=0pt,belowskip=0pt}

\gdef\thefigure{\arabic{figure}}
\gdef\thetable{\arabic{table}}
\usepackage{amsmath}

\RequirePackage{hyperref}

\tikzstyle{abstractbox} = [draw=black, fill=white, rectangle,
inner sep=10pt, style=rounded corners=5pt, drop shadow={fill=black,
opacity=0.2}]
\newcommand{\boxabstractd}[3][]{%
\par\medskip
\begin{center}
  \begin{tikzpicture}
    \node [abstractbox, #1,inner sep=10pt] (box)
    {\begin{minipage}{#2}
         #3%\footnotesize
      \end{minipage}};
  \end{tikzpicture}
\end{center}}

%===================================================
\lstset{%
  frame=shadowbox,framexleftmargin=7mm, xleftmargin=7mm, xrightmargin=1mm, framexrightmargin=0mm,framextopmargin=0mm,belowcaptionskip=0.5\baselineskip,
  rulesepcolor=\color{gray!65},
	rulecolor=\color{gray},
	numbersep=3mm,%2
    numbers=left,%
    numberstyle=\color{codmatlab}\fontfamily{phv}\tiny,
	breaklines=true,breakatwhitespace=true}

\lstdefinestyle{sycoddpython}{%
    belowcaptionskip=0.5\baselineskip,
	framexleftmargin=1mm,
	xleftmargin=2mm,
	inputencoding=utf8/latin1,
	inputencoding=utf8,
	numbers=left,%none
	numberstyle=\color{codmatlab}\fontfamily{phv}\tiny,%
	stepnumber=1,%
	numbersep=10pt,%
	tabsize=2,%
%	escapeinside={(*@}{@*)},
%	breakatwhitespace=true,
%	backgroundcolor=\color{green!1},
	showstringspaces=false,
	showspaces=false,
	showtabs=false,
	language=Python,
%	basicstyle=\ttfamily\footnotesize,
%	basicstyle={\lstbasicfont},
	basicstyle=\fontencoding{T1}\fontfamily{fvm}\footnotesize\selectfont,%\ttfamily
%	basicstyle={\fontsize{9pt}{11pt}\usefont{T1}{fvm}{m}{n}\selectfont},
%	otherkeywords={self},
	keywordstyle=\color{azultwo}\bfseries,
	keywordstyle= [2]\color{ContorTeo}, % just to check that it works
	emph={MyClass,__init__,AbrirArchivo,VerArchivo},
	emphstyle=\color{morado02},
	emph={[2]from,import,pass,return},
	emphstyle={[2]\color[HTML]{DD52F0}},
	emph={[3]range},
	emphstyle={[3]\color[HTML]{D17032}},
	emph={[4]for,in,def},
	emphstyle={[4]\color{blue}},
	columns=fullflexible,
	prebreak = \mbox{{\color{gray}\tiny$\searrow$}},%
	breaklines=true,
	%inputencoding=utf8,
	extendedchars=true,
	literate=%
      {á}{{\'a}}1  {é}{{\'e}}1  {í}{{\ \@'i}}1 {ó}{{\'o}}1  {ú}{{\'u}}1
      {Á}{{\'A}}1  {É}{{\'E}}1  {Í}{{\'I}}1 {Ó}{{\'O}}1  {Ú}{{\'U}}1
      {à}{{\`a}}1  {è}{{\`e}}1  {ì}{{\ \@`i}}1 {ò}{{\`o}}1  {ù}{{\`u}}1
      {À}{{\`A}}1  {È}{{\'E}}1  {Ì}{{\`I}}1 {Ò}{{\`O}}1  {Ù}{{\`U}}1
      {ä}{{\"a}}1  {ë}{{\"e}}1  {ï}{{\"\@i}}1 {ö}{{\"o}}1  {ü}{{\"u}}1
      {Ä}{{\"A}}1  {Ë}{{\"E}}1  {Ï}{{\"I}}1 {Ö}{{\"O}}1  {Ü}{{\"U}}1
      {â}{{\^a}}1  {ê}{{\^e}}1  {î}{{\ \@^i}}1 {ô}{{\^o}}1  {û}{{\^u}}1
      {Â}{{\^A}}1  {Ê}{{\^E}}1  {Î}{{\^I}}1 {Ô}{{\^O}}1  {Û}{{\^U}}1
      {œ}{{\oe}}1  {Œ}{{\OE}}1  {æ}{{\ae}}1 {Æ}{{\AE}}1  {ß}{{\ss}}1
      {ç}{{\c c}}1 {Ç}{{\c C}}1 {ø}{{\o}}1  {Ø}{{\O}}1   {å}{{\r a}}1
      {Å}{{\r A}}1 {ã}{{\~a}}1  {õ}{{\~o}}1 {Ã}{{\~A}}1  {Õ}{{\~O}}1
      {ñ}{{\~n}}1  {Ñ}{{\~N}}1  {¿}{{?`}}1  {¡}{{!`}}1
      {°}{{\textdegree}}1 {º}{{\textordmasculine}}1 {ª}{{\textordfeminine}}1,
	commentstyle=\color{grverd},
	stringstyle=\color{Sec1},
}

%--------------------------------------------------------
\makeatletter
\def\@copyrightyear{\number\the\year}
\def\@university{UNIVERSIDAD NACIONAL DE SAN CRISTÓBAL DE HUAMANGA}
\def\@faculty{Facultad de Ingeniería de Minas, Geología y Civil}
\def\@dept{Ingeniería de Sistemas}
\def\@teacher{Ing. AbdulRahman Mossad Bashit}
\def\@course{Nueve Li Nueve DX\,-\,9li9}
\def\@tema{Nueve Li Nueve}
\def\@email{}
\newcommand{\copyrightyear}[1]{\renewcommand{\@copyrightyear}{#1}}
\newcommand{\university}[1]{\renewcommand{\@university}{#1}}
\newcommand{\faculty}[1]{\renewcommand{\@faculty}{#1}}
\newcommand{\dept}[1]{\renewcommand{\@dept}{#1}}
\newcommand{\teacher}[1]{\renewcommand{\@teacher}{#1}}
\newcommand{\course}[1]{\renewcommand{\@course}{#1}}
\newcommand{\tema}[1]{\renewcommand{\@tema}{#1}}
\newcommand{\dytema}{\@tema}
\newcommand{\email}[1]{\renewcommand{\@email}{#1}}
\newcommand{\dyemail}{\@email}

\def\@author{Nueve Li Nueve}
\def\@title{\empty}
\newcommand{\dycopyrightyear}{\@copyrightyear}
\newcommand{\dyuniversity}{\@university}
\newcommand{\dyfaculty}{\@faculty}
\newcommand{\dydept}{\@dept}
\newcommand{\dycourse}{\@course}
\newcommand{\dyauthor}{\@author}
\newcommand{\dytitle}{\@title}
\newcommand{\dyteacher}{\@teacher}
\newcommand{\@academicyear}{2015- 2016}
\newcommand{\academicyear}[1]{\renewcommand{\@academicyear}{#1}}
\newcommand{\dyacademicyear}{\@academicyear}
\newcommand{\dyyear}{\number\year}
\makeatother
%===================================================
%--------------------------------------------------------
%para tablas
\usepackage{array,ragged2e}
\newcolumntype{L}[1]{>{\hspace{0pt}\RaggedRight}m{#1}}
\newcolumntype{C}[1]{>{\hspace{0pt}\Centering}m{#1}}
\newcolumntype{R}[1]{>{\hspace{0pt}\RaggedLeft}m{#1}}

%----------------------------------------------------------------
%-------------------------------------
\usepackage{tcolorbox}
\newcommand{\probbkn}[6][0cm]{%
\par\medskip\hspace{#1}
\begin{tikzpicture}[font=\bfseries, every node/.style={signal,%auto,
 signal to=nowhere},remember picture,overlay]%
 \fill[color=#2!10](0cm,0.3cm) rectangle (15.5cm,-0.26cm);
 \fill[color=#2](0cm,0.3cm) rectangle (15.5cm,0.28cm);
 \node[anchor=north west,fill=#2, signal to=east, inner sep=3pt, text=white, drop shadow,line width=1.2pt, draw = #2] at (-0.1,0.3) {\makebox[#3][l]%[l]
 {\hspace{#4}\,#5}};
  \node[text=azzul,line width=0pt] at (10cm,0cm) {\makebox[10.5cm][r]%[l]
 {\normalfont#6}};
\end{tikzpicture}
\par\medskip}
%\probbkn[espacio]{color}{ancho}{espacio}{numero}{libro}\vspace{-5pt}

\newcommand{\nwpar}{\par\vspace{0.3\baselineskip}}

\newcommand\linsecsol[2][-4pt]{\par\vspace{#1}
{\setlength{\parskip}{0pt}\color{#2}\rule[0.5cm]{\textwidth}{1.1pt}}\par\vspace{-0.7cm}}

\newcommand\linseccmdf[3][-4pt]{\par\vspace{#1}
{\color{#3}\rule[0.5cm]{\textwidth}{1.1pt}}\par\vspace{#2}}

\newcommand\sysolution[2][azultwo]{\nwpar%
{\setlength{\parskip}{0pt}\bfseries\color{#1}Solución}\hfill\setlength{\parskip}{0pt}\linseccmdf[0pt]{-0.5cm}{cyan (process)}}

\newcommand\sysolutiont[2][azultwo]{\nwpar%
{\setlength{\parskip}{0pt}\bfseries\color{#1}Solución}\hfill\linsecsol[0pt]{azzul}}

%----------------------------------------------------------------------------------------
%------------------------------- Cuadro de notas ------------------------------
\tcbset{%
rghstyle/.style={boxsep=1mm,left=1mm,right=1mm,top=1mm,center,before=\par\bigskip,%\centering
    after=\par\medskip,
    breakable,fontupper=\setlength{\parskip}{8pt plus 1pt minus 1pt},
    freelance,
    boxrule=1pt,
% Código para las cajas intactas:
    frame code={%
    \path[draw=\contorCodeNote, line width=0.6pt,fill=\contorCodeNote]
        %([yshift=-7.5pt]frame.north west) -- ([xshift=7.5pt]frame.north west) --
        (frame.north west)-- (frame.north west)--
        (frame.north east)-- (frame.north east|-interior.north east)--
        (frame.north west|-interior.north west)-- cycle;
	%\path[draw=\contorCodeNote!30, line width=20pt] ([xshift=2.25cm,yshift=-10.3pt]frame.north west) --%
	%([xshift=-0.3pt,yshift=-10.3pt]frame.north east);
    },
    interior titled code={
    \path[draw=\contorCodeNote, line width=0.6pt,fill=\fondoCodeNote]
        (frame.west|-interior.north west)-- (frame.east|-interior.north east)--
        (frame.east|-interior.south east)-- (frame.west|-interior.south west)-- cycle;
    \path[anchor=north west,draw=white, line width=1.5pt] ([xshift=-1pt]frame.west|-interior.north west) --%
	([xshift=1pt]frame.east|-interior.north east);
    },
% código para la primera parte de una secuencia de ruptura:
    skin first is subskin of={emptyfirst}{%emptyfirst
    frame code={%
    \path[draw=\contorCodeNote, line width=0.6pt,fill=\contorCodeNote]
        (frame.north west)-- (frame.north west)--
        (frame.north east)-- (frame.north east|-interior.north east)--
        (frame.north west|-interior.north west)-- cycle;
    %\path[draw=\fondoCodeNote,line width=20pt] ([xshift=2.2cm,yshift=-10.3pt]frame.north west) --%
	%([xshift=-0.3pt,yshift=-10.3pt]frame.north east);
	%\path[draw=\contorCodeNote!30, line width=20pt] ([xshift=2.25cm,yshift=-10.3pt]frame.north west) --%
	%([xshift=-0.3pt,yshift=-10.3pt]frame.north east);
    },
    interior titled code={
    \path[draw=\contorCodeNote, line width=0.6pt,fill=\fondoCodeNote]
        (frame.west|-interior.north west)-- (frame.east|-interior.north east)--
        (frame.east|-interior.south east)-- (frame.west|-interior.south west)-- cycle;
    \path[anchor=north west,draw=white, line width=1.5pt] ([xshift=-1pt]frame.west|-interior.north west) --%
	([xshift=1pt]frame.east|-interior.north east);
    },},
% código para la parte media de una secuencia de ruptura:
    skin middle is subskin of={emptymiddle}{%
    frame code={\path[draw=\contorCodeNote, line width=0.6pt, fill=\fondoCodeNote] (frame.south west)--
      (frame.north west)-- (frame.north east)--
      (frame.south east)-- (frame.south east)--cycle;
      },
      },
% código de la última parte de una secuencia de ruptura:
    skin last is subskin of={emptylast}{%
    frame code={\path[draw=\contorCodeNote, line width=0.6pt, fill=\fondoCodeNote] (frame.south west)--
      (frame.north west)-- (frame.north east)--
      (frame.south east)-- (frame.south east)--cycle;
      },
    },
%------------
    fonttitle=\bfseries\vphantom{dy},%
    %fontupper=\sffamily,
},
notehibbstyle/.style={boxsep=1mm,left=1mm,right=1mm,top=1mm,center,before=\par\bigskip,%\centering
    breakable,fontupper=\setlength{\parskip}{8pt plus 1pt minus 1pt},
    freelance,
    boxrule=1pt,
    width=0.98\linewidth,
% Código para las cajas intactas:
    frame code={%
    \path[draw=\contorCodeNote, line width=0.6pt,fill=\contorCodeNote]%\contorCodeNote
        %([yshift=-7.5pt]frame.north west) -- ([xshift=7.5pt]frame.north west) --
        (frame.north west)-- (frame.north west)--
        (frame.north east)-- (frame.north east|-interior.north east)--
        (frame.north west|-interior.north west)-- cycle;
	\path[draw=\fondoCodeNote,line width=20pt] ([xshift=0.65cm,yshift=-10.3pt]frame.north west) --%
	([xshift=-0.3pt,yshift=-10.3pt]frame.north east);
	\path[draw=\contorCodeNote!60, line width=20pt] ([xshift=0.7cm,yshift=-10.3pt]frame.north west) --%
	([xshift=-0.3pt,yshift=-10.3pt]frame.north east);
    },
    interior titled code={
    \path[draw=\contorCodeNote, line width=0.6pt,fill=\fondoCodeNote]
        (frame.west|-interior.north west)-- (frame.east|-interior.north east)--
        (frame.east|-interior.south east)-- (frame.west|-interior.south west)-- cycle;
    \path[draw=white, line width=1.5pt] ([xshift=-1pt]frame.west|-interior.north west) --%
	([xshift=1pt]frame.east|-interior.north east);
    },
% código para la primera parte de una secuencia de ruptura:
    skin first is subskin of={emptyfirst}{%
    frame code={%
    \path[draw=\contorCodeNote, line width=0.6pt,fill=\contorCodeNote]
        (frame.north west)-- (frame.north west)--
        (frame.north east)-- (frame.north east|-interior.north east)--
        (frame.north west|-interior.north west)-- cycle;
	\path[draw=\fondoCodeNote,line width=20pt] ([xshift=0.65cm,yshift=-10.3pt]frame.north west) --%
	([xshift=-0.3pt,yshift=-10.3pt]frame.north east);
	\path[draw=\contorCodeNote!60, line width=20pt] ([xshift=0.7cm,yshift=-10.3pt]frame.north west) --%
	([xshift=-0.3pt,yshift=-10.3pt]frame.north east);
    },
    interior titled code={
    \path[draw=\contorCodeNote, line width=0.6pt,fill=\fondoCodeNote]
        (frame.west|-interior.north west)-- (frame.east|-interior.north east)--
        (frame.east|-interior.south east)-- (frame.west|-interior.south west)-- cycle;
    \path[draw=white, line width=1.5pt] ([xshift=-1pt]frame.west|-interior.north west) --%
	([xshift=1pt]frame.east|-interior.north east);
    },},
% código para la parte media de una secuencia de ruptura:
    skin middle is subskin of={emptymiddle}{%
    frame code={\path[draw=\contorCodeNote, line width=0.6pt, fill=\fondoCodeNote] (frame.south west)--
      (frame.north west)-- (frame.north east)--
      (frame.south east)-- (frame.south east)--cycle;
      },
      },
% código de la última parte de una secuencia de ruptura:
    skin last is subskin of={emptylast}{%
    frame code={\path[draw=\contorCodeNote, line width=0.6pt, fill=\fondoCodeNote] (frame.south west)--
      (frame.north west)-- (frame.north east)--
      (frame.south east)-- (frame.south east)--cycle;
      },
    },
    fonttitle=\bfseries\vphantom{dy}
},
notehibbnbstyle/.style={boxsep=1mm,left=1mm,right=1mm,top=1mm,center,before=\par\bigskip,%\centering
    breakable,fontupper=\setlength{\parskip}{8pt plus 1pt minus 1pt},
    freelance,
    boxrule=1pt,
    width=0.98\linewidth,
% Código para las cajas intactas:
    frame code={%
    \path[draw=\contorCodeNote, line width=0.6pt,fill=\contorCodeNote]%\contorCodeNote
        %([yshift=-7.5pt]frame.north west) -- ([xshift=7.5pt]frame.north west) --
        (frame.north west)-- (frame.north west)--
        (frame.north east)-- (frame.north east|-interior.north east)--
        (frame.north west|-interior.north west)-- cycle;
	\path[draw=\fondoCodeNote,line width=20pt] ([xshift=0.65cm,yshift=-10.3pt]frame.north west) --%
	([xshift=-0.3pt,yshift=-10.3pt]frame.north east);
	\path[draw=\contorCodeNote!60, line width=20pt] ([xshift=0.7cm,yshift=-10.3pt]frame.north west) --%
	([xshift=-0.3pt,yshift=-10.3pt]frame.north east);
    },
    interior titled code={
    \path[draw=\contorCodeNote, line width=0.6pt,fill=\fondoCodeNote]
        (frame.west|-interior.north west)-- (frame.east|-interior.north east)--
        (frame.east|-interior.south east)-- (frame.west|-interior.south west)-- cycle;
    \path[draw=white, line width=1.5pt] ([xshift=-1pt]frame.west|-interior.north west) --%
	([xshift=1pt]frame.east|-interior.north east);
    },
% código para la primera parte de una secuencia de ruptura:
    skin first is subskin of={emptyfirst}{%
    frame code={%
    \path[draw=\contorCodeNote, line width=0.6pt,fill=\contorCodeNote]
        (frame.north west)-- (frame.north west)--
        (frame.north east)-- (frame.north east|-interior.north east)--
        (frame.north west|-interior.north west)-- cycle;
    \path[draw=\fondoCodeNote,line width=20pt] ([xshift=0.65cm,yshift=-10.3pt]frame.north west) --%
	([xshift=-0.3pt,yshift=-10.3pt]frame.north east);
	\path[draw=\contorCodeNote!60, line width=20pt] ([xshift=0.7cm,yshift=-10.3pt]frame.north west) --%
	([xshift=-0.3pt,yshift=-10.3pt]frame.north east);
    },
    interior titled code={
    \path[draw=\contorCodeNote, line width=0.6pt,fill=\fondoCodeNote]
        (frame.west|-interior.north west)-- (frame.east|-interior.north east)--
        (frame.east|-interior.south east)-- (frame.west|-interior.south west)-- cycle;
    \path[draw=white, line width=1.5pt] ([xshift=-1pt]frame.west|-interior.north west) --%
	([xshift=1pt]frame.east|-interior.north east);
    },},
% código para la parte media de una secuencia de ruptura:
    skin middle is subskin of={emptymiddle}{%
    frame code={\path[draw=\contorCodeNote, line width=0.6pt, fill=\fondoCodeNote] (frame.south west)--
      (frame.north west)-- (frame.north east)--
      (frame.south east)-- (frame.south east)--cycle;
      },
      },
% código de la última parte de una secuencia de ruptura:
    skin last is subskin of={emptylast}{%
    frame code={\path[draw=\contorCodeNote, line width=0.6pt, fill=\fondoCodeNote] (frame.south west)--
      (frame.north west)-- (frame.north east)--
      (frame.south east)-- (frame.south east)--cycle;
      },
    },
    fonttitle=\bfseries\vphantom{dy}
},
notehibbextremstyle/.style={boxsep=1mm,left=1mm,right=1mm,top=1mm,flush right,before=\par\bigskip,%\hfill
    breakable,fontupper=\setlength{\parskip}{8pt plus 1pt minus 1pt},
    freelance,
    boxrule=1pt,
    width=0.98\linewidth,
% Código para las cajas intactas:
    frame code={%
    \path[draw=\contorCodeNote, line width=0.6pt,fill=\contorCodeNote]%\contorCodeNote
        %([yshift=-7.5pt]frame.north west) -- ([xshift=7.5pt]frame.north west) --
        (frame.north west)-- (frame.north west)--
        (frame.north east)-- (frame.north east|-interior.north east)--
        (frame.north west|-interior.north west)-- cycle;
	\path[draw=\fondoCodeNote,line width=20pt] ([xshift=0.65cm,yshift=-10.3pt]frame.north west) --%
	([xshift=-0.3pt,yshift=-10.3pt]frame.north east);
	\path[draw=\contorCodeNote!60, line width=20pt] ([xshift=0.7cm,yshift=-10.3pt]frame.north west) --%
	([xshift=-0.3pt,yshift=-10.3pt]frame.north east);
    },
    interior titled code={
    \path[draw=\contorCodeNote, line width=0.6pt,fill=\fondoCodeNote]
        (frame.west|-interior.north west)-- (frame.east|-interior.north east)--
        (frame.east|-interior.south east)-- (frame.west|-interior.south west)-- cycle;
    \path[draw=white, line width=1.5pt] ([xshift=0.302pt]frame.west|-interior.north west) --%
	([xshift=-0.302pt]frame.east|-interior.north east);
    },
% código para la primera parte de una secuencia de ruptura:
    skin first is subskin of={emptyfirst}{%
    frame code={%
    \path[draw=\contorCodeNote, line width=0.6pt,fill=\contorCodeNote]
        (frame.north west)-- (frame.north west)--
        (frame.north east)-- (frame.north east|-interior.north east)--
        (frame.north west|-interior.north west)-- cycle;
	\path[draw=\fondoCodeNote,line width=20pt] ([xshift=0.65cm,yshift=-10.3pt]frame.north west) --%
	([xshift=-0.3pt,yshift=-10.3pt]frame.north east);
	\path[draw=\contorCodeNote!60, line width=20pt] ([xshift=0.7cm,yshift=-10.3pt]frame.north west) --%
	([xshift=-0.3pt,yshift=-10.3pt]frame.north east);
    },
    interior titled code={
    \path[draw=\contorCodeNote, line width=0.6pt,fill=\fondoCodeNote]
        (frame.west|-interior.north west)-- (frame.east|-interior.north east)--
        (frame.east|-interior.south east)-- (frame.west|-interior.south west)-- cycle;
    \path[draw=white, line width=1.5pt] ([xshift=0.302pt]frame.west|-interior.north west) --%
	([xshift=-0.302pt]frame.east|-interior.north east);
    },},
% código para la parte media de una secuencia de ruptura:
    skin middle is subskin of={emptymiddle}{%
    frame code={\path[draw=\contorCodeNote, line width=0.6pt, fill=\fondoCodeNote] (frame.south west)--
      (frame.north west)-- (frame.north east)--
      (frame.south east)-- (frame.south east)--cycle;
      },
      },
% código de la última parte de una secuencia de ruptura:
    skin last is subskin of={emptylast}{%
    frame code={\path[draw=\contorCodeNote, line width=0.6pt, fill=\fondoCodeNote] (frame.south west)--
      (frame.north west)-- (frame.north east)--
      (frame.south east)-- (frame.south east)--cycle;
      },
    },
    fonttitle=\bfseries\vphantom{dy}
},
notehibbcantstyle/.style={boxsep=1mm,left=1mm,right=1mm,top=1mm,before=\par\bigskip,
    breakable,fontupper=\setlength{\parskip}{8pt plus 1pt minus 1pt},
    freelance,
    boxrule=1pt,
    width=0.98\linewidth,
% Código para las cajas intactas:
    frame code={%
    \path[draw=\contorCodeNote, line width=0.6pt,fill=\contorCodeNote]%\contorCodeNote
        %([yshift=-7.5pt]frame.north west) -- ([xshift=7.5pt]frame.north west) --
        (frame.north west)-- (frame.north west)--
        (frame.north east)-- (frame.north east|-interior.north east)--
        (frame.north west|-interior.north west)-- cycle;
	\path[draw=\fondoCodeNote,line width=20pt] ([xshift=0.65cm,yshift=-10.3pt]frame.north west) --%
	([xshift=-0.3pt,yshift=-10.3pt]frame.north east);
	\path[draw=\contorCodeNote!60, line width=20pt] ([xshift=0.7cm,yshift=-10.3pt]frame.north west) --%
	([xshift=-0.3pt,yshift=-10.3pt]frame.north east);
    },
    interior titled code={
    \path[draw=\contorCodeNote, line width=0.6pt,fill=\fondoCodeNote]
        (frame.west|-interior.north west)-- (frame.east|-interior.north east)--
        (frame.east|-interior.south east)-- (frame.west|-interior.south west)-- cycle;
    \path[draw=white, line width=1.5pt] ([xshift=0.302pt]frame.west|-interior.north west) --%
	([xshift=-0.302pt]frame.east|-interior.north east);
    },
% código para la primera parte de una secuencia de ruptura:
    skin first is subskin of={emptyfirst}{%
    frame code={%
    \path[draw=\contorCodeNote, line width=0.6pt,fill=\contorCodeNote]
        (frame.north west)-- (frame.north west)--
        (frame.north east)-- (frame.north east|-interior.north east)--
        (frame.north west|-interior.north west)-- cycle;
	\path[draw=\fondoCodeNote,line width=20pt] ([xshift=0.65cm,yshift=-10.3pt]frame.north west) --%
	([xshift=-0.3pt,yshift=-10.3pt]frame.north east);
	\path[draw=\contorCodeNote!60, line width=20pt] ([xshift=0.7cm,yshift=-10.3pt]frame.north west) --%
	([xshift=-0.3pt,yshift=-10.3pt]frame.north east);
    },
    interior titled code={
    \path[draw=\contorCodeNote, line width=0.6pt,fill=\fondoCodeNote]
        (frame.west|-interior.north west)-- (frame.east|-interior.north east)--
        (frame.east|-interior.south east)-- (frame.west|-interior.south west)-- cycle;
    \path[draw=white, line width=1.5pt] ([xshift=0.302pt]frame.west|-interior.north west) --%
	([xshift=-0.302pt]frame.east|-interior.north east);
    },},
% código para la parte media de una secuencia de ruptura:
    skin middle is subskin of={emptymiddle}{%
    frame code={\path[draw=\contorCodeNote, line width=0.6pt, fill=\fondoCodeNote] (frame.south west)--
      (frame.north west)-- (frame.north east)--
      (frame.south east)-- (frame.south east)--cycle;
      },
      },
% código de la última parte de una secuencia de ruptura:
    skin last is subskin of={emptylast}{%
    frame code={\path[draw=\contorCodeNote, line width=0.6pt, fill=\fondoCodeNote] (frame.south west)--
      (frame.north west)-- (frame.north east)--
      (frame.south east)-- (frame.south east)--cycle;
      },
    },
    fonttitle=\bfseries\vphantom{dy}
}
}

\newtcolorbox{drlsy}[2][0.98\linewidth]{rghstyle,width=#1,title=#2}
\newtcolorbox{dsynote}[2][0.7\linewidth]{notehibbstyle,width=#1,title=\hspace{0.6cm}#2}
\newtcolorbox{dsynotecz}[2][0.98\linewidth]{notehibbnbstyle,width=#1,title=\hspace{0.6cm}#2}
\newtcolorbox{dsynoterz}[2][0.5\linewidth]{notehibbextremstyle,width=#1,title=\hspace{0.6cm}#2}
\newtcolorbox{dsynotelz}[2][0.5\linewidth]{notehibbcantstyle,width=#1,title=\hspace{0.6cm}#2}

\newtcolorbox{syfigurelrz}[3][\textwidth]{
%sidebyside,
%sidebyside align=top,
width=#1,
%lefthand width=#2,
%sidebyside gap=12pt,
drop shadow,
%freelance,
fontupper=\setlength{\parskip}{8pt plus 1pt minus 1pt},
%lower separated=true,
boxrule=1pt, boxsep=1mm, left=1mm, right=1mm, top=1mm, enhanced,
title=\vphantom{gÍ}#2, fonttitle=\bfseries, colback=#3!5!white, colframe=#3, center, before=\par\medskip,%\centering
}

\newtcolorbox{dyNote}[2][azzul]{colback=#1!5,
frame hidden, boxrule=0pt, boxsep=0pt, breakable,fontupper=\setlength{\parskip}{8pt plus 1pt minus 1pt},
enlarge bottom by=0.3cm, enhanced jigsaw,
borderline west={3pt}{0pt}{#1},
title={#2\\[1mm]}, colbacktitle={#1},
coltitle={#1}, coltext={black},
fonttitle={\bfseries\vphantom{dy}}, attach title to upper={}}%

\makeatletter
\tcbset{%mytheorem/.code
  sytheorm/.code args={#1#2#3#4#5}{
    \refstepcounter{#2}\label{#4}
    \pgfkeysalso{title={\setbox\z@=\hbox{#1\,\,\,\,\,\,\,\,\,\,{\color{#5}\ifnum\csname the#2\endcsname<10 0\fi\csname the#2\endcsname\ }}%
	\hangindent\wd\z@\hangafter=1 \mbox{#1\,\,\,\,\,\,\,\,\,\,{\color{#5}\ifnum\csname the#2\endcsname<10 0\fi\csname the#2\endcsname\ }}\,\,{\color{black}#3}}}},%
}

\newcommand{\mtcbmaketheorem}[5]{
  \newtcolorbox{#1}[4][\linewidth]{#3,sytheorm={#2}{#4}{##3}{#5:##4}{##2},width=##1}
}
\makeatother

\newcounter{nej}

\tcbset{
defffstyle/.style={boxsep=1mm,left=1mm,right=1mm,top=1mm,center,before=\par\bigskip,%\centering
    breakable,
    freelance,
    boxrule=1pt,
%**    width=0.98\linewidth,
% Código para las cajas intactas:
    frame code={%
    \path[draw=\contorCodeExerc, line width=0.6pt,fill=\contorCodeExerc]
        %([yshift=-7.5pt]frame.north west) -- ([xshift=7.5pt]frame.north west) --
        (frame.north west)-- (frame.north west)--
        (frame.north east)-- (frame.north east|-interior.north east)--
        (frame.north west|-interior.north west)-- cycle;
	\path[draw=\fondoCodeExerc,line width=20pt] ([xshift=2.25cm,yshift=-10.3pt]frame.north west) --%
	([xshift=-0.3pt,yshift=-10.3pt]frame.north east);
	\path[draw=\contorCodeExerc!30, line width=20pt] ([xshift=2.3cm,yshift=-10.3pt]frame.north west) --%
	([xshift=-0.3pt,yshift=-10.3pt]frame.north east);
    },
    interior titled code={
    \path[draw=\contorCodeExerc, line width=0.6pt,fill=\fondoCodeExerc]
        (frame.west|-interior.north west)-- (frame.east|-interior.north east)--
        (frame.east|-interior.south east)-- (frame.west|-interior.south west)-- cycle;
    \path[draw=white, line width=1.5pt] ([xshift=-1pt]frame.west|-interior.north west) --%
	([xshift=1pt]frame.east|-interior.north east);
    },
% código para la primera parte de una secuencia de ruptura:
    skin first is subskin of={emptyfirst}{%
    frame code={%
    \path[draw=\contorCodeExerc, line width=0.6pt,fill=\contorCodeExerc]
        (frame.north west)-- (frame.north west)--
        (frame.north east)-- (frame.north east|-interior.north east)--
        (frame.north west|-interior.north west)-- cycle;
    \path[draw=\fondoCodeExerc,line width=20pt] ([xshift=2.25cm,yshift=-10.3pt]frame.north west) --%
	([xshift=-0.3pt,yshift=-10.3pt]frame.north east);
	\path[draw=\contorCodeExerc!30, line width=20pt] ([xshift=2.3cm,yshift=-10.3pt]frame.north west) --%
	([xshift=-0.3pt,yshift=-10.3pt]frame.north east);
    },
    interior titled code={
    \path[draw=\contorCodeExerc, line width=0.6pt,fill=\fondoCodeExerc]
        (frame.west|-interior.north west)-- (frame.east|-interior.north east)--
        (frame.east|-interior.south east)-- (frame.west|-interior.south west)-- cycle;
    \path[draw=white, line width=1.5pt] ([xshift=-1pt]frame.west|-interior.north west) --%
	([xshift=1pt]frame.east|-interior.north east);
    },},
% código para la parte media de una secuencia de ruptura:
    skin middle is subskin of={emptymiddle}{%
    frame code={\path[draw=\contorCodeExerc, line width=0.6pt, fill=\fondoCodeExerc] (frame.south west)--
      (frame.north west)-- (frame.north east)--
      (frame.south east)-- (frame.south east)--cycle;
      },
      },
% código de la última parte de una secuencia de ruptura:
    skin last is subskin of={emptylast}{%
    frame code={\path[draw=\contorCodeExerc, line width=0.6pt, fill=\fondoCodeExerc] (frame.south west)--
      (frame.north west)-- (frame.north east)--
      (frame.south east)-- (frame.south east)--cycle;
      },
    },
    fonttitle=\bfseries%\sffamily
},
}

\mtcbmaketheorem{Exercise}{Ejercicio}{defffstyle}{nej}{ej}

%-------------------------------
\newtcolorbox{syNoteB}[5][\textwidth]{
  breakable,width=#1,before=\par\vskip15pt,after=\par\medskip,
  enhanced,boxsep=1mm,left=1mm,right=1mm,top=4mm,boxrule=1pt,
  colframe=#4,frame style={drop shadow},drop shadow=white,%boxed title style={drop shadow},
  colback=#4!5,fontupper=\setlength{\parskip}{8pt plus 1pt minus 1pt},
  overlay unbroken and first={
   \node[fill=#3,
      rounded corners,
      draw=#2,
      text=black,
      line width=0.5pt,
      inner sep=3pt,
      anchor=west,
      xshift=12pt]
   at (frame.north west){\bfseries #5};
  }
}

\newtcolorbox{syNoteBz}[5][\textwidth]{
  breakable,width=#1,before=\par\vskip15pt,after=\par\medskip,
  enhanced,boxsep=1mm,left=1mm,right=1mm,top=4mm,boxrule=1pt,
  colframe=#4,%frame style={drop shadow},drop shadow=white,boxed title style={drop shadow},
  colback=#4!5,fontupper=\setlength{\parskip}{8pt plus 1pt minus 1pt},
  overlay unbroken and first={
   \node[fill=#3,
      rounded corners,
      draw=#2,
      text=black,
      line width=0.5pt,
      inner sep=3pt,
      anchor=west,
      xshift=12pt]
   at (frame.north west){\bfseries #5};
  }
}
%-------------------------------
%-------------------------------

\makeatletter
\renewcommand{\boxed}[1]{%\textcolor{\boxcolor}{%
\tikz[baseline={([yshift=-1ex]current bounding box.center)}]%
\node[rectangle, minimum width=1ex,rounded corners,draw=azzul,fill=azzul!10, inner sep=6pt]{\normalcolor\m@th$\displaystyle#1$};}
\makeatother

%-------------------------------
\newcommand*{\itemcirccz}[2]{%
\scriptsize\protect\tikz[baseline=-3pt]%
\protect\node[draw=#1,line width=1pt, inner sep=1.5pt, circle,fill=#1!25!white]{\color{#1}\bf#2};}
\setlist[itemize,1]{label=\leavevmode\hbox{to 1.2ex{\hss\vrule height .9ex width .7ex depth -.2ex\hss}}}

\newcommand{\sdconditions}[4][\textwidth-\pgfkeysvalueof{/pgf/inner xsep}-4mm]{%
\begin{tikzpicture}
\node[text width=#1, align=justify, inner sep=0mm, outer sep=0] (one)
{\parbox[t]{\textwidth}{\phantom{dy}}};
\node[anchor=north west, draw={trueblue}, fill={trueblue!7}, text=black, thin, rectangle, rounded corners] (bcfg) at (one.north west) {#3};%style=rounded corners=5pt,
\node[anchor=north west, text width=#1, align=center, inner sep=0mm, outer sep=0] (fig) at ([xshift=0pt, yshift=-4pt]bcfg.south west) {\parbox[t]{\textwidth}{#4}};
\path[draw=#2]
    ($(bcfg.west)+(0pt,0pt)$) [sharp corners] --
    ($(bcfg.west)+(-6pt,0pt)$) --
    ($(fig.south west)+(-6pt,0pt)$) -- ($(fig.south west)+(0pt,0pt)$) ;
\end{tikzpicture}
}

\newtcolorbox{dyNoteImportant}[4][azzul]{
  enhanced,frame empty, boxsep=2pt, boxrule=1pt,left=2mm,right=2mm,top=2mm, arc=0pt, auto outer arc,%,
  colback=#2,
  coltitle=#3,
  title=\bfseries #4,
  attach boxed title to top left={xshift=0mm},
  boxed title style={empty},
  underlay boxed title={%\titlepath
  \fill[#1]
  (title.south east)-- (title.east) coordinate (A) to[curve to,out=90,in=0] ($(A)+(-5mm,5mm)$)-- ($(title.north west)+(5mm,0mm)$) coordinate (B) to[curve to,out=180,in=90] ($(B)+(-5mm,-5mm)$) coordinate (C)-- ($(C)+(0mm,-5mm)$) to[curve to,out=90,in=180] ($(title.south west)+(+5mm,0mm)$) coordinate (F) --cycle;
      \draw[#1,ultra thick]
      ([yshift=.5\pgflinewidth]title.south east)--
      ([yshift=.5\pgflinewidth]title.south-|interior.east);
}
}
%---------------------------------------------------------------------------------------------------------------
